\documentclass[12pt]{report}
  \usepackage[english]{babel}
  \usepackage[utf8]{inputenc}
  \usepackage[dvipsnames]{xcolor}
  \usepackage{hyperref}
  \usepackage{listings}
  \usepackage{parcolumns}
  \usepackage{algorithm}
  \usepackage{algorithmicx}
  \usepackage{algpseudocode}
  \usepackage{enumitem}
  \usepackage{geometry}
  \usepackage{soul}
  \usepackage{graphicx}
  \usepackage{enumitem}
  \usepackage{csquotes}
  \usepackage{bookmark}
  \usepackage{mdframed}
  \usepackage{mathtools}
  \usepackage{amsmath}
  \usepackage{amsthm}
  \usepackage[toc]{appendix}
  \usepackage[
    backend=biber,%
    style=apa%
  ]{biblatex}

  \colorlet{light-gray}{gray!10}

  % Bibliography Setup
  \addbibresource{main.bib}
  \newcommand{\CiteSection}[2]{%
    \parencite[~\S{#2}]{#1}
  }

  % Theorem Environments
  \theoremstyle{definition}
  \newtheorem*{defn*}{Definition}
  \theoremstyle{plain}
  \newtheorem*{equ*}{Equation}
  \theoremstyle{plain}
  \newtheorem*{examp*}{Example}

  % Definitions for Algorithmic Environments
  \algdef{SE}[VARIABLES]{GVariables}{EndGVariables}
    {\algorithmicvariables}
    {\algorithmicend\ \algorithmicvariables}
  \algnewcommand{\algorithmicvariables}{\textbf{global variables}}

  \algdef{SE}[VARIABLES]{LVariables}{EndLVariables}
    {\algorithmiclvariables}
    {\algorithmicend\ \algorithmiclvariables}
  \algnewcommand{\algorithmiclvariables}{\textit{local variables}}

  \renewcommand{\algorithmicrequire}{\textbf{Input:}}
  \renewcommand{\algorithmicensure}{\textbf{Output:}}
  \renewcommand\thealgorithm{}

  % Settings for math-mode
  \makeatletter
  \def\mathcolor#1#{\@mathcolor{#1}}
  \def\@mathcolor#1#2#3{%
    \protect\leavevmode
    \begingroup
      \color#1{#2}#3%
    \endgroup
  }
  \makeatother

  % Image Directory
  \graphicspath{ {screenshots/} }
  % Hyperlink Setup
  \hypersetup{
    colorlinks = true,
    urlcolor = blue,
    linkcolor = blue,
    citecolor = blue
  }

  % Syntax-Highlight for Code Snippets
  %% UTF-8 Support
  \lstset{literate=%
    {á}{{\'a}}1 {é}{{\'e}}1 {í}{{\'i}}1 {ó}{{\'o}}1 {ú}{{\'u}}1
    {Á}{{\'A}}1 {É}{{\'E}}1 {Í}{{\'I}}1 {Ó}{{\'O}}1 {Ú}{{\'U}}1
    {à}{{\`a}}1 {è}{{\`e}}1 {ì}{{\`i}}1 {ò}{{\`o}}1 {ù}{{\`u}}1
    {À}{{\`A}}1 {È}{{\'E}}1 {Ì}{{\`I}}1 {Ò}{{\`O}}1 {Ù}{{\`U}}1
    {ä}{{\"a}}1 {ë}{{\"e}}1 {ï}{{\"i}}1 {ö}{{\"o}}1 {ü}{{\"u}}1
    {Ä}{{\"A}}1 {Ë}{{\"E}}1 {Ï}{{\"I}}1 {Ö}{{\"O}}1 {Ü}{{\"U}}1
    {â}{{\^a}}1 {ê}{{\^e}}1 {î}{{\^i}}1 {ô}{{\^o}}1 {û}{{\^u}}1
    {Â}{{\^A}}1 {Ê}{{\^E}}1 {Î}{{\^I}}1 {Ô}{{\^O}}1 {Û}{{\^U}}1
    {Ã}{{\~A}}1 {ã}{{\~a}}1 {Õ}{{\~O}}1 {õ}{{\~o}}1
    {œ}{{\oe}}1 {Œ}{{\OE}}1 {æ}{{\ae}}1 {Æ}{{\AE}}1 {ß}{{\ss}}1
    {ű}{{\H{u}}}1 {Ű}{{\H{U}}}1 {ő}{{\H{o}}}1 {Ő}{{\H{O}}}1
    {ç}{{\c c}}1 {Ç}{{\c C}}1 {ø}{{\o}}1 {å}{{\r a}}1 {Å}{{\r A}}1
    {€}{{\euro}}1 {£}{{\pounds}}1 {«}{{\guillemotleft}}1
    {»}{{\guillemotright}}1 {ñ}{{\~n}}1 {Ñ}{{\~N}}1 {¿}{{?`}}1
  }
  %% Appearance
  \lstset{
    breaklines=true,
    captionpos=b,
    frame=tblr,
    tabsize=2,
    numbers=left,
    showstringspaces=false,
    commentstyle=\color{Red},
    keywordstyle=\color{Violet},
    stringstyle=\color{OliveGreen},
    backgroundcolor=\color{light-gray}
  }

  % Page and Text Layout
  \geometry{%
    a4paper,%
    top=1in,%
    bottom=1in,%
    left=1in,%
    right=1in%
  }
  \setlength{\headheight}{15pt}

  \lstMakeShortInline[language=c++,columns=fixed]|

  \newenvironment{ldefinitions}
    {\left.\begin{aligned}}
    {\end{aligned}\right\rbrace}

  \title{
    Module 5 Concept Discussion\\%
    \Large{Coding Standards}
  }
  \author{Ashton Hellwig}
  \date{\today}

\begin{document}
  \maketitle
  \tableofcontents
%   \lstlistoflistings
  \newpage


  \chapter{Initial Post}
    \vspace*{-10pt}
    \section*{Research Prompt}
      \begin{mdframed}[backgroundcolor=green!20]
        Many groups use standard conventions for programming, including the
          ordering of public and private members in the class. See Examples
          10-3 through 10-5 for one such convention. What is the advantage of
          using such a standard? Do you have your own preferences?
      \end{mdframed}
    \vspace{-15pt}
    \section{Response}
      \subsection{Introduction}
        First, so no one will have to open the book and view Examples 10.3-10.5,
          I will include a summary (including the examples) here given in the
          \S 10.8 of the textbook \autocite{malik2015}.
        \begin{lstlisting}[language=c++,caption={Example 10.3},label={e3}]
class clockType {
public:
  void setTime(int, int, int);
  void getTime(int&, int&, int&) const;
  void printTime() const;
  void incrementSeconds();
  void incrementMinutes();
  void incrementHours();
  bool equalTime(const clockType&) const;

private:
  int hour;
  int minute;
  int second;
}
        \end{lstlisting}
        \begin{lstlisting}[language=c++,caption={Example 10.4},label={e4}]
class clockType {
private:
  int hour;
  int minute;
  int second;

public:
  void setTime(int, int, int);
  void getTime(int&, int&, int&) const;
  void printTime() const;
  void incrementSeconds();
  void incrementMinutes();
  void incrementHours();
  bool equalTime(const clockType&) const;
}
        \end{lstlisting}
        \begin{lstlisting}[language=c++,caption={Example 10.5},label={e5}]
class clockType {
  int hour;
  int minute;
  int second;

public:
  void setTime(int, int, int);
  void getTime(int&, int&, int&) const;
  void printTime() const;
  void incrementSeconds();
  void incrementMinutes();
  void incrementHours();
  bool equalTime(const clockType&) const;
}
        \end{lstlisting}

        Initially, the most notable differences are in that in Listing
          \ref{e3} the private members are placed after the public members,
          while Listing \ref{e4} places private members first and Listing
          \ref{e5} places private members first, without an access specifier.

        The book states that without an access specifier, members default to
          private and it is common practice to list \textbf{public} members
          before \textbf{private} members in order to ``focus your attention
          on the public members'' \autocite{malik2015}. I find this odd,
          considering in nearly every editor I have tried (My main is
          Visual Studio Code), the C++17 snippets for classes almost always
          put private members first. I actually believe it makes more sense
          to place private members first, since you tend to know the
          \textit{types} and \textit{pieces} of data you need prior to
          writing its functions. Also, it is only in a |class| where the default
          access specifier is |private|, |struct|s default to a public access specifier
          and can be adjusted by setting |private| as the access specifier.

        In order to discuss programming standards and conventions (as well as
          my preferred method of style and convention), we must go over some of
          the more common programming style-guides/conventions and contrast
          their motivations.

        Since the book discusses the order of members within a standard class, even
          though there are \textit{ample} differing opinions on style and convention, we will
          be focusing on the declaration and access of member variables.

        \subsection{CPP Core Guidelines} 
          \subsubsection{Links}
            \begin{itemize}
              \item \href{https://github.com/isocpp/CppCoreGuidelines}{Homepage}
              \item \href{https://github.com/isocpp/CppCoreGuidelines/blob/master/CppCoreGuidelines.md}{Guidelines}
            \end{itemize}
          \subsubsection{Highlights}
          
        \subsection{My Personally Preferred Style}
          \subsubsection{Links}
            \begin{itemize}
                \item \href{https://github.com/ashellwig/ashellwig_m4c8_programming_assignment}{Repository Example 1}
                \item \href{https://github.com/ashellwig/ashellwig_m5c9_programming_assignment}{Repository Example 2}
            \end{itemize}
          \subsubsection{Highlights}
            I personally use a style based on LLVM`s, configured through the use of a development
              utility which comes with clang called `\href{https://clang.llvm.org/docs/ClangFormat.html}{clang-format}'. You can view my
              personal configuration file I use for (roughly) any of my C++ based projects
              \href{https://github.com/ashellwig/ashellwig_m5c9_programming_assignment/blob/master/.clang-format}{here}. I will also include below a code snippet which shows what the style looks like for
              different language components and features.
              
            \begin{lstlisting}[language=c++,caption={Ashton`s clang-format Result -- Macros}]
namespace ashwig_debug {
/**
 * @brief Prints the output of a function call when it is sent to stderr rather
 * than stdout. Use for debugging assistance.
 */
#define verbose_print(message, ...)                                            \
  std::cerr << "[" << __FILE__ << "]: "                                        \
            << "[" << __PRETTY_FUNCTION__ << "]: "                             \
            << "[line:" << __LINE__ << "] >> " << message << __VA_ARGS__       \
            << std::endl;
} // namespace ashwig_debug
\end{lstlisting} 

            \begin{lstlisting}[language=c++,caption={Ashton`s clang-format Result -- Macros}]
/**
 * @file chapter8.hh
 * @author Ashton Scott Hellwig (ahellwig@student.cccs.edu)
 * @brief This file contains the prototypes used for
 * calculating the vote statistics of data input by the user.
 * @date 2020-04-16
 *
 * Assignment: Module 4 Chapter 8 Programming Assignment.
 * Description: This file contains the prototypes used for
 * calculating the vote statistics of data input by the user.
 * Instructor: Jeffrey Hemmes.
 * Course: [CSC 160] Introduction to Programming (C++).
 * Date: 16 April 2020.
 */

#ifndef _CHAPTER8_HH_INCLUDED
#define _CHAPTER8_HH_INCLUDED
#include <string>

#ifndef NDEBUG
#  define DEBUG 1
#  include "ashwig_debug.hh"
#else
#  define DEBUG 0
#endif

#include "ashwig_exceptions.hh"

namespace chapter8 {

class Candidate {
public:
  // Static constants
  static const int m_numberOfCandidates =
      5; //*< Number of candidates in election.

  // Constructors
  Candidate(); //*< Construct class with user input via std::cin.
  explicit Candidate(
      std::string); //*< Construct class with one string(for testing).

  // Setters
  void setTotalVotes(int[]);
  int calculatePercentOfVotes();
  // Getters
  int getTotalVotes() const;
  int getWinnerIndex() const;
  void getUserInput(std::string[], int[]);
  // Printers
  void printResult() const;

protected:
  std::string m_names[m_numberOfCandidates]; //*< Last names of candidates.
  int m_votes[m_numberOfCandidates]; //*< Number of votes for each candidate.
  double m_percentOfVotes[m_numberOfCandidates]; //*< Percentage of total votes
                                                 //*< for each candidate.
  int m_totalVotes; //*< Total votes across all candidates.
  // clang-format disable
}; // class Candidate
// clang-format enable
} // namespace chapter8
#endif // !CHAPTER8_HH_INCLUDED
}
\end{lstlisting}

            \begin{lstlisting}[language=c++,caption={Ashton`s clang-format Result -- Macros}]
/**
 * @file chapter8.cxx
 * @author Ashton Scott Hellwig (ahellwig@student.cccs.edu)
 * @brief This file contains the function implementations used for
 * calculating the vote statistics of data input by the user.
 * @date 2020-04-21
 *
 * Assignment: Module 4 Chapter 8 Programming Assignment.
 * Description: This file contains the function implementations used for
 * calculating the vote statistics of data input by the user.
 * Instructor: Jeffrey Hemmes.
 * Course: [CSC 160] Introduction to Programming (C++).
 * Date: 21 April 2020.
 */

#include "../include/chapter8.hh"
#include <iomanip>  //std::setw, std::setprecision
#include <iostream> // std::cin, std::cout
#include <sstream>  // std::stringstream
#include <string>   // std::string

/**
 * @brief Construct a new chapter8::Candidate::Candidate object
 */
chapter8::Candidate::Candidate() {
  // Prompt user for input to initialize our parallel arrays.
  getUserInput(m_names, m_votes);

  // Calculate number of total votes (sum of votes for all candidates).
  for (int iter = 0; iter < m_numberOfCandidates; iter++) {
    m_totalVotes += m_votes[iter];
  }

  // Calculate the percentage of total votes received by each candidate and
  // store the value in a separate array.
  for (int iter = 0; iter < m_numberOfCandidates; iter++) {
    m_percentOfVotes[iter] =
        100.00 * (m_votes[iter] / static_cast<double>(m_totalVotes));
  }
}

/**
 * @brief Construct a new chapter8::Candidate::Candidate object
 *
 * @param inputString The last names and number of votes for each candidates in
 * one string.
 */
chapter8::Candidate::Candidate(std::string inputString) {
  std::stringstream userInputStream(inputString);

  // Initialize Arrays
  for (int iter = 0; iter < m_numberOfCandidates; iter++) {
    userInputStream >> m_names[iter] >> m_votes[iter];
  }

  // Calculate number of total votes (sum of votes for all candidates).
  for (int iter = 0; iter < m_numberOfCandidates; iter++) {
    m_totalVotes += m_votes[iter];
  }

  // Calculate the percentage of total votes received by each candidate and
  // store the value in a separate array.
  for (int iter = 0; iter < m_numberOfCandidates; iter++) {
    m_percentOfVotes[iter] = m_votes[iter] / static_cast<double>(m_totalVotes);
  }
}

/**
 * @brief Sets the Candidate class's private member variables using input
 * obtained via stdin.
 *
 * @param names The last names of the candidates in the election.
 * @param votes The number of votes received by the candidate.
 */
void chapter8::Candidate::getUserInput(std::string names[], int votes[]) {
  // Prompt User
  std::cout
      << "Enter candidate's last name and the votes received by the candidate."
      << std::endl;

  // Initialize our arrays
  for (int iter = 0; iter < m_numberOfCandidates; iter++) {
    std::cin >> names[iter] >> votes[iter];
  }
}

/**
 * @brief Calculates the sum of votes received by all candidates.
 *
 * @return int The sum of votes received by all candidates.
 */
int chapter8::Candidate::getTotalVotes() const { return m_totalVotes; }

/**
 * @brief Returns the index of the candidate with the most votes.
 *
 * @return int The index of the candidate with the most votes.
 */
int chapter8::Candidate::getWinnerIndex() const {
  int mostVotes = 0; //*< The number of votes received by the winning candidate.
  int index = 0;     //*< Index of array.
  int lcv = sizeof(m_votes) / sizeof(m_votes[0]); //*< Loop Control Variable.

  for (int i = 0; i < lcv; i++) {
    if (m_votes[i] > mostVotes) {
      mostVotes = m_votes[i];
      index = i;
    }
  }

  return index;
}

/**
 * @brief Generates the desired output for the Chapter 8 Programming Assignment.
 */
void chapter8::Candidate::printResult() const {
  // Set local variables.
  int lcv = sizeof(m_votes) / sizeof(m_votes[0]); //*< Loop control variable.

  // Output formatted header
  std::cout << std::setprecision(3) << std::fixed;
  std::cout << "Candidate" << std::setw(4) << " "
            << "Votes Received" << std::setw(4) << " "
            << "\% of Total Votes" << std::endl;

  // Loop which prints the election results based on our user's input.
  for (int i = 0; i < lcv; i++) {
    std::cout << std::fixed << std::setprecision(2) << std::left;
    std::cout << std::setw(16) << m_names[i] << std::setw(23) << m_votes[i]
              << m_percentOfVotes[i] << '\n';
  }

  // Find our election winner.
  int indexOfWinner = getWinnerIndex();

  // Output total number of votes received by ALL candidates.
  std::cout << "Total"
            << " " << std::setw(15) << m_totalVotes << std::endl;

  // Output winner of election.
  std::cout << "The winner of the election is " << m_names[indexOfWinner] << "."
            << std::endl;
}
\end{lstlisting}
          
 
        \subsection{LLVM}
          \subsubsection{Links}
            \begin{itemize}
                \item \href{https://llvm.org/docs/CodingStandards.html}{Guidelines}
            \end{itemize}
          \subsubsection{Highlights}
            Going back to the ordering of members within classes, the LLVM style guide has
              different ideas, assuming that examples are sufficient evidence as to the order of declaration,
              as it does not explicitly discuss the issue. Private members are listed first, and public
              members are after.\footnote{\href{https://llvm.org/docs/CodingStandards.html\#use-of-class-and-struct-keywords}{LLVM: Use of class and struct Keywords}}
              
            In addition to this, LLVM has a few other notable style conventions. I will go over a few of them below.
              \begin{enumerate}
                  \item |#include <iostream>| is forbidden:
                  \begin{itemize}
                      \item This stream implementation inherently injects a static constructor into each
                        instance. This breaks LLVM`s rule concerning \href{https://llvm.org/docs/CodingStandards.html#static-constructor}{static constructors}.
                        LLVM uses its own implementation, \texttt{raw\_ostream}, for this. 
                  \end{itemize}
                  \item Avoid the use of \lstinline[language=c++,columns=fixed]{std::endl}
                    \begin{itemize}
                        \item |std::endl| causes a flush to the output
                          stream, in addition to a new line. Generally this is not the desired effect. It is
                          better to use \lstinline[language=c++,columns=fixed]{std::cout << "Hello!\n";} or
                          \lstinline[language=c++,columns=fixed]{std::cout << "Hello!" << '\n';} to end a line
                          when the output stream does not need to be flushed.
                    \end{itemize}
              \end{enumerate}
              

        \subsection{Google}
          \subsubsection{Links}
            \begin{enumerate}
                \item \href{https://google.github.io/styleguide/cppguide.html}{Homepage}
            \end{enumerate}
          \subsubsection{Highlights}
            Unlike most style guides, Google`s explicitly discusses the declaration order
              of class members.\footnote{\href{https://google.github.io/styleguide/cppguide.html\#Declaration_Order}{Google Style Guide (C++): Declaration Order}}
              Google states to place \texttt{public:} members first, then \texttt{protected:}, and then
              finally \texttt{private:} members. It does not specify whether or not the word ``private''
              is required for the declaration of private member variables.
              
            Google mainly differs from other style guides in its \href{https://google.github.io/styleguide/cppguide.html\#Naming}{naming conventions}, specifically
              \href{https://google.github.io/styleguide/cppguide.html#Variable_Names}{with variable names}.
              
            Mozilla`s style guidelines are derived from Google`s.

        \subsection{Microsoft} 
          Placeholder.
        
        \subsection{Advantages to Using a Standard}
          There are numerous advantages to the adherence to a standard convention of style.
            Not only in programming/development, but in any collaborative work environment.
            For example, even in the marijuana dispensary I currently work in we constantly
            have to use/write/revise our \textbf{SOP}s -- \textit{Standard Operating Procedures}.
            SOPs are designed to help employees perform complicated tasks consistently in the
            same way, and with the level of regulation put towards our industry it is incredibly
            important to rememver the minor details of every task. In programming, code style
            guidelines and defining contributor guidelines is the best way to ensure that
            code stays consistent and readable by everyone collaborating on the project and to make
            the code base and repository easier to maintain as wekk as decrease its
            computational complexity and run-time.


  % Replies to Other`s Responses
  %% ! TEX root=../main.tex

\newpage
\chapter{Responses}

  \section{Response 1}
    \begin{mdframed}[backgroundcolor=green!20]
      Reply to \textbf{} (\textit{Post ID: 00000000})

      Placeholder.
    \end{mdframed}
    Placeholder.


  \section{Response 2}
    \begin{mdframed}[backgroundcolor=green!20]
      Reply to \textbf{} (\textit{Post ID: 00000000})

      Placeholder.
    \end{mdframed}
    Placeholder.



  % Bibliography
  \newpage
  \nocite{*}
  \printbibliography[
    heading=bibintoc,
    title={Bibliography}
  ]
\end{document}
